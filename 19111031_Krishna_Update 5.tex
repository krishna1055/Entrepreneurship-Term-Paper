\documentclass{article}

\usepackage[margin=1.5in]{geometry}
\usepackage[dvipsnames]{xcolor}

\title{Market Research for "Sadabahar"}
\author{Krishna Agrawal, 19111031}
\date{\today}

\begin{document}
\maketitle


\section{Introduction}

In this term paper I will be doing Market Research for our venture Sadabahar. I will also try to come up with a Business Model for our venture.

Sadabahar is a one stop solution for all Indoor Plant Lovers. We provide complete support for Indoor Plants. Our services don't end when a customer buys a plant, it just starts from there. 
We accompany our customers in the journey of growing their plant and help them in keeping their plant alive Sadabahar with them! 
 
Benefits of Indoor Plants:

The indoor plants improve the oxygen level, keep the house air pure, remove pollutants, and also decrease the rate of indoor air pollution. They also lifts one’s mood, boosts creativity and reduces stress and depression.


\section{Indoor Plants Market Scope and Market Size}

The market for indoor plants in India is estimated at over Rs 250 crores annually and is growing at 15% to 20%.

Rising air pollution concerns in not only the metros but in smaller towns across the country has made indoor plant a much-needed solution for people, leading them to add more green inside their homes. Also, this has led to increased awareness of indoor plants.

The indoor plants market is segmented on the basis of type, application and product. 

\begin{enumerate}

\item On the basis of types, the indoor plants market is segmented into shade- loving plants, low light plants and high light plants.
\item On the basis of application, the indoor plants market is segmented into absorb harmful gases and home decoration.
\item On the basis of product, the indoor plants market is segmented into succulent plants, berbaceous plants, woody plants and hydroponic plants.
\end{enumerate}

\section{Business Model}

Currently we are still figuring out, though we are looking for:

\begin{itemize}

\item Aggregator Business Model:

Aggregator Business Model is a network model where the aggregator firm collects information about particular offering providers, sign contracts with such providers, and sell their services under its own brand.

Since the aggregator is a brand, it provides an offering that has uniform quality and price, even though it is offered by different partner providers.

The offering providers never become aggregator’s employees and continue to be the owners of the product or service provided. Aggregator just helps them in marketing in a unique win-win manner.

For us it will mean that we will be having a platform which will have products listed by us by partnering with a single or only a bunch of supplier, basically selling the plants under our brand name.

We can try this model for our plant based venture but for now its chances seems less.

\item Marketplace business model:

A marketplace is a platform where buyers and sellers interact and trade through a network.

The online marketplace doesn’t own any products or services. It simply acts as a mediator and connects buyers and sellers.

In other words, a marketplace is a platform that provides value to both buyers and sellers. Sellers get a place to sell their products and grow their brand while buyers are provided with multiple choices and high-quality products all in one place.

For us it will mean that we will be having a platform which will have multiple plant sellers and buyers choosing from the listed products.

We are more interested in this model, though we are figuring out how can we become an essential part of our partners business and develop a unique selling point.

\item Personal Brand

Like doing everything on our own, growing plants on our own, maintaining them on our own. 

This is the least possible thing.

\end{itemize}

\section{Actual Ground situation}

I along with my team visited more than 5 nurseries this week, and came to know that 4 of them didn't grew indoor plants on their own and just bought them from places like Pune, Kolkata and Andhra Pradesh. Only 1 nursery owner claimed to grow the plants in raipur itself though we are not sure how sure her claim is. 

So now we are figuring out how to procure plants for our venture. For now we are sure that we will be able to find even one nursery that grow the plant by itself in raipur and partner with it.


\section{Customer Insights}

In this week we have conducted surveys from people, and are currently analysing them to create a report. Once we have covered it, i will include it here.

\end{document}