\documentclass{article}

\usepackage[margin=1.5in]{geometry}
\usepackage[dvipsnames]{xcolor}

\title{Market Research for "Sadabahar"}
\author{Krishna Agrawal, 19111031}
\date{\today}

\begin{document}
\maketitle


\section{Introduction}

In this term paper I will be doing Market Research for our venture Sadabahar. I will also try to come up with a Business Model for our venture.

Sadabahar is a one stop solution for all Indoor Plant Lovers. We provide complete support for Indoor Plants. Our services don't end when a customer buys a plant, it just starts from there. 
We accompany our customers in the journey of growing their plant and help them in keeping their plant alive Sadabahar with them! 
 
Benefits of Indoor Plants:

The indoor plants improve the oxygen level, keep the house air pure, remove pollutants, and also decrease the rate of indoor air pollution. They also lifts one’s mood, boosts creativity and reduces stress and depression.


\section{Indoor Plants Market Scope and Market Size}

The market for indoor plants in India is estimated at over Rs 250 crores annually and is growing at 15% to 20%.

Rising air pollution concerns in not only the metros but in smaller towns across the country has made indoor plant a much-needed solution for people, leading them to add more green inside their homes. Also, this has led to increased awareness of indoor plants.

The indoor plants market is segmented on the basis of type, application and product. 

\begin{enumerate}

\item On the basis of types, the indoor plants market is segmented into shade- loving plants, low light plants and high light plants.
\item On the basis of application, the indoor plants market is segmented into absorb harmful gases and home decoration.
\item On the basis of product, the indoor plants market is segmented into succulent plants, berbaceous plants, woody plants and hydroponic plants.
\end{enumerate}

\end{document}